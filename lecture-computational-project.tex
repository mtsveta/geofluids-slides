
\section{Computational project}
%
% https://ethz.zoom.us/rec/play/eff8FsJ-ZfoiA-l2EFNn_wY9RPhwZQ-N86KuVVsMc6Eb9UmYvAYoNhZkmpU7_mdvpgb37KsFzb6X9W9J.KNuq2Acqbg1fZbAy?continueMode=true
\begin{frame}{Quiz on standard potentials}
	
	\begin{itemize}
		\item \alert{\textbf{Quiz:}} what can we do alternatively to taking reference values of $\mu_0$, if we want to find standard potentials for pressures and temperatures lying in between the reference values?
		%
		\begin{center}
			\href{http://etc.ch/vR5e}{\textcolor{indigo(dye)}{\tt http://etc.ch/vR5e}} \quad or \quad 
			\includegraphics[height=0.2\columnwidth]{figures/computational-projects/polls.png}
		\end{center}
		%
		\hiddenpause
		\vskip 10pt
		\item {\textbf{Answer:}} we interpolate between neighboring values of standard potentials.   
		
	\end{itemize}
\end{frame}
%
\begin{frame}[fragile]{Quiz on variable of EquilibriumState providing information on the solubility}
	
\begin{lstlisting}[language=Python]
class EquilibriumState:
    def __init__(self):
        self.T = None  # temperature in K
        self.P = None  # pressure in Pa
        self.n = None  # amounts of species in mol
        self.m = None  # masses of species in g
        self.y = None  # Lagrange multipliers y in J/mol
        self.z = None  # Lagrange multipliers z in J/mol
        self.c = None  # concentrations of the species
        self.a = None  # activities of the species
        self.g = None  # activity coefficients of the species
\end{lstlisting}

\begin{itemize}
	\item \alert{\textbf{Quiz:}} out of all these properties stored in the class {\bf EquilibriumState}, which one will tell us the solubility of gas or mineral? 
	%
	%\vskip 10pt
	\begin{center}
		\href{http://etc.ch/vR5e}{\textcolor{indigo(dye)}{\tt http://etc.ch/vR5e}} \quad or \quad 
		\includegraphics[height=0.08\columnwidth]{figures/computational-projects/polls.png}
	\end{center}
	%
	%\vskip 10pt
	\hiddenpause
	\vskip 10pt
	\item {\textbf{Answer:}} concentrations of the species, c (molal).
% 	\item {\textbf{Answer:}} to make sure that we reach its solubility point, to dissolve the {\bf maximum} amount of CO$_2$(g) in the brine.
\end{itemize}
	
\end{frame}
%
\begin{frame}{Quiz on adding NaCl into the recipe}
	
	\begin{itemize}
		\item \alert{\textbf{Quiz:}} what does happen to the solubility of CO$_2$(g) if we add NaCl into the recipe?
		%
		\vskip 10pt
		\begin{center}
			\href{http://etc.ch/vR5e}{\textcolor{indigo(dye)}{\tt http://etc.ch/vR5e}} \quad or \quad 
			\includegraphics[height=0.2\columnwidth]{figures/computational-projects/polls.png}
		\end{center}
		%
		\vskip 10pt
		\hiddenpause
		\item {\textbf{Answer:}} it decreases, salting-out effect.
	\end{itemize}
	
\end{frame}
%
\begin{frame}{Quiz on the amount of CO$_2$ amount}
		
		\begin{itemize}
			\item \alert{\textbf{Quiz:}} why do we add so much initial CO$_2$(g)?
			%
			\vskip 10pt
			\begin{center}
				\href{http://etc.ch/vR5e}{\textcolor{indigo(dye)}{\tt http://etc.ch/vR5e}} \quad or \quad 
				\includegraphics[height=0.2\columnwidth]{figures/computational-projects/polls.png}
			\end{center}
			%
			\vskip 10pt
			\hiddenpause
			\vskip 10pt
			\item {\textbf{Answer:}} to make sure that we reach its solubility point, to dissolve the {\bf maximum} amount of CO$_2$(g) in the brine.
		\end{itemize}
	
\end{frame}
\begin{frame}{Acidity of recipes}
	\small
	\begin{itemize}
		\item Given three mixture: \\
			\begin{itemize}
				\item  {\bf Mixture 1} :  1kg of water, 0.6 mol of NaCl, 1 mol of calcite \\
				\item  {\bf Mixture 2} :  1kg of water, 0.6 mol of NaCl, 1 mol of calcite, 5 mol of CO$_2$(g)\\
				\item  {\bf Mixture 3} :  1kg of water, 5 mol of CO$_2$(g)\\
			\end{itemize}
		\item \alert{\textbf{Quiz:}} sort out these mixture w.r.t increasing acidity (decreasing pH).\\[5pt]
		%
		\vskip 5pt
		\begin{center}
			\href{http://etc.ch/vR5e}{\textcolor{indigo(dye)}{\tt http://etc.ch/vR5e}} \quad or \quad 
			\includegraphics[height=0.18\columnwidth]{figures/computational-projects/polls.png}
		\end{center}
		%
		\hiddenpause
		\vskip 5pt
		\item {\textbf{Answer:}} 1, 2, 3 because \\
		\begin{itemize}
			\item pH (Mixture 1) $\approx$ 10 \\
			\item pH (Mixture 2) $\approx$ 6 \\
			\item pH (Mixture 3) $\approx$ 4 \\
		\end{itemize}
	\end{itemize}

	%b = element_amounts(kgH2O=1, molCO2=0.0, molNaCl=0.6, molCaCO3=1.0) -> pH = 10 (alkaline) 
	%b = element_amounts(kgH2O=1, molCO2=5.0, molNaCl=0.6, molCaCO3=1.0) -> pH = 6 (slightly acidic)
	%b = element_amounts(kgH2O=1, molCO2=5.0, molNaCl=0.0, molCaCO3=0.0) -> pH = 3.9 (acidic)
\end{frame}

\begin{frame}{pH dependence on the added amount of added CO$_2$(g)}
	\small
	\begin{itemize}

		\item \alert{\textbf{Quiz:}} what happens around the point of added CO$_2$(g) equal 0.8 mols?\\[5pt]
		%
		\vskip 10pt
		\begin{center}
			\href{http://etc.ch/vR5e}{\textcolor{indigo(dye)}{\tt http://etc.ch/vR5e}} \quad or \quad 
			\includegraphics[height=0.18\columnwidth]{figures/computational-projects/polls.png}
		\end{center}
		%
		\vskip 10pt
		\hiddenpause
		\vskip 10pt
		\item {\textbf{Answer:}} we have reached the saturation point of CO$_2$(g) in water, i.e., the aqueous solution cannot dissolve more CO$_2$ gas.
		\end{itemize}
\end{frame}

\begin{frame}{Calcite solubility dependence on the added CO$_2$(g)}
	\small
	\begin{itemize}
		
		\item \alert{\textbf{Quiz:}} what does this plato with dissolved calcite indicate? \\[5pt]
		%
		\vskip 10pt
		\begin{center}
			\href{http://etc.ch/vR5e}{\textcolor{indigo(dye)}{\tt http://etc.ch/vR5e}} \quad or \quad 
			\includegraphics[height=0.18\columnwidth]{figures/computational-projects/polls.png}
		\end{center}
		%
		\vskip 10pt
		\hiddenpause
		\vskip 10pt
		\item {\textbf{Answer:}} we have reached the saturation point of CO$_2$(g) in water, which does not effect the solubility of calcite.
	\end{itemize}
\end{frame}