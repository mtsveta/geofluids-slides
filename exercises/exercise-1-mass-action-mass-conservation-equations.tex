%% LyX 2.3.4.2 created this file.  For more info, see http://www.lyx.org/.
%% Do not edit unless you really know what you are doing.
\documentclass[12pt,english]{article}
\usepackage{mathpazo}
\usepackage{tgcursor}
\usepackage[T1]{fontenc}
\usepackage[latin9]{inputenc}
\usepackage{geometry}
\geometry{verbose,tmargin=2cm,bmargin=2cm,lmargin=2cm,rmargin=2cm}
\setlength{\parskip}{\smallskipamount}
\setlength{\parindent}{0pt}
\synctex=-1
\usepackage{babel}
\usepackage{booktabs}
\usepackage{amsmath}
\usepackage{amssymb}
\usepackage[authoryear]{natbib}
\usepackage[unicode=true,
 bookmarks=true,bookmarksnumbered=false,bookmarksopen=false,
 breaklinks=false,pdfborder={0 0 0},pdfborderstyle={},backref=false,colorlinks=false]
 {hyperref}
\hypersetup{
 pdfauthor={Svetlana Kyas}}

\makeatletter

%%%%%%%%%%%%%%%%%%%%%%%%%%%%%% LyX specific LaTeX commands.
%% Because html converters don't know tabularnewline
\providecommand{\tabularnewline}{\\}

\makeatother

\begin{document}
\title{\textbf{Exercise 1} \textbf{(Geofluids -- Part III)}\emph{}\\
\emph{Mass action and mass conservation equations and their use for
chemical equilibrium calculations}}
\author{Lecturer: Dr. Svetlana Kyas}

\maketitle

Consider the following chemical species in three different phases,
and the chemical elements that compose them:
\begin{center}
\begin{tabular}{ll}
\toprule 
\textbf{Index} & \textbf{Species}\tabularnewline
\midrule
1 & {$\mathrm{CO_{2}(aq)}$}\tabularnewline
2 & {$\mathrm{CO_{3}^{-2}(aq)}$}\tabularnewline
3 & {$\mathrm{Cl^{-}(aq)}$}\tabularnewline
4 & {$\mathrm{H^{+}(aq)}$}\tabularnewline
5 & {$\mathrm{H_{2}O(l)}$}\tabularnewline
6 & {$\mathrm{HCO_{3}^{-}(aq)}$}\tabularnewline
7 & {$\mathrm{Na^{+}(aq)}$}\tabularnewline
8 & {$\mathrm{OH^{-}(aq)}$}\tabularnewline
9 & {$\mathrm{CO_{2}(g)}$}\tabularnewline
10 & {$\mathrm{NaCl(s)}$}\tabularnewline
\bottomrule
\end{tabular}\hspace{2cm}%
\begin{tabular}{ll}
\toprule 
\textbf{Index} & \textbf{Element}\tabularnewline
\midrule
1 & C\tabularnewline
2 & Cl\tabularnewline
3 & H\tabularnewline
4 & Na\tabularnewline
5 & Ol\tabularnewline
\bottomrule
\end{tabular}
\par\end{center}

The following system of linearly independent reactions can be written
for this chemical system:
\begin{align*}
\mathrm{H_{2}O(aq)} & \mathrm{\rightleftharpoons H^{+}(aq)+OH^{-}(aq)}\tag{1}\\
\mathrm{HCO_{3}^{-}(aq)+H^{+}(aq)} & \mathrm{\rightleftharpoons CO_{2}(aq)+H_{2}O(aq)}\tag{2}\\
\mathrm{H_{2}O(aq)+CO_{2}(aq)} & \mathrm{\rightleftharpoons CO_{3}^{-2}(aq)+2H^{+}(aq)}\tag{3}\\
\mathrm{CO_{2}(g)} & \mathrm{\rightleftharpoons CO_{2}(aq)}\tag{4}\\
\mathrm{NaCl(s)} & \mathrm{\rightleftharpoons Na^{+}(aq)+Cl^{-}(aq)}\tag{5}
\end{align*}

\newpage{}

\section*{Exercise}
\begin{itemize}
\item[a)] Write the \textbf{system of mass action equations }for these reactions.
\item[b)] Write the \textbf{system of mass conservation equations }for the
elements.
\item[c)] Discuss whether the combination of these systems of equations are
enough to compute $n=(n_{1},\ldots,n_{10})$ and, if so, what kind
of algorithm would you use for this.
\end{itemize}
\textbf{Considerations:}
\begin{itemize}
\item Use $n_{1}$, ..., $n_{10}$ to denote the amounts of the species
(\emph{unknown});
\item Use $b_{1}$, ..., $b_{5}$ to denote the amounts of the elements
(\emph{known});
\item Use $a_{1}$, ..., $a_{10}$ to denote the activities of the species
(\emph{unknown, function of }$n$);
\item Use $K_{1}$, ..., $K_{5}$ to denote the equilibrium constants of
the reactions (\emph{known}).
\end{itemize}
\textbf{Remarks:}
\begin{itemize}
\item In the next classes, we will see how activities, $a_{i}$, are calculated
for aqueous, gaseous and mineral species. For now, know that activities
are non-linear functions that depend on the amounts of the species,
$n=(n_{1},\ldots,n_{10})$, temperature $T$ and pressure $P$. Thus,
$a_{i}=a_{i}(n;T,P)$, where $T$ and $P$ are given and $n$ is what
we want to find. 
\item To write the mass conservation equations, determine the chemical formula
coefficients $A_{ji}$ below:
\[
\sum_{i=1}^{10}A_{ji}n_{i}=b_{j}\qquad(j=1,\ldots,5).
\]
For example, $A_{\mathrm{O,HCO_{3}^{-}}}\equiv A_{\mathrm{2,6}}=3$.
\item To write the mass action equations, use the general mass action equation:
\[
K_{m}=\frac{\prod_{p}a_{p}^{\nu_{p}}}{\prod_{r}a_{r}^{\nu_{r}}}\qquad(m=1,\ldots,5),
\]
for the reaction written in general form:
\[
\sum_{r}\nu_{r}A_{r}\rightleftharpoons\sum_{p}\nu_{p}A_{p}.
\]
For example, for reaction (1), we have:
\[
K_{1}=\frac{a_{2}a_{3}}{a_{1}}.
\]
\end{itemize}

\end{document}
